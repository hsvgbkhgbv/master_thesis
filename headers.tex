\documentclass{book}
\usepackage[margin=1.3in,bottom=1in,top=1in]{geometry}
\usepackage{fancyhdr}

\pagestyle{fancy}
\usepackage[sorting = none,
            doi = true  %lesedato for url-adresse
            ]{biblatex} %none gir bibliografi i sitert rekkefølge
\addbibresource{reference.bib}



\begin{document}
\title{Masteroppgave}
\author{Vegard Solberg}
 
\maketitle
\tableofcontents


\chapter{Norway}
Det skal vel la seg gjøre med litt tekst før vi starter med seksjonene

\section{nisselandet i nord}
Norge er et parlamentarisk demokrati og konstitusjonelt monarki, hvor Harald V siden 1991 er konge og statsoverhode, og Erna Solberg (H) er siden 2013 statsminister. Norge er en enhetsstat, med to administrative nivå under staten: fylker og kommuner. Den samiske delen av befolkningen har, gjennom Sametinget og Finnmarksloven, til en viss grad selvstyre og innflytelse over tradisjonelt samiske områder. Selv om Norge har avvist medlemskap i Den europeiske union gjennom to folkeavstemninger, har Norge gjennom EØS-avtalen tette bånd til unionen, og gjennom NATO med USA. Norge er en betydelig bidragsyter i De forente nasjoner (FN), og har deltatt med soldater i flere utenlandsoperasjoner med mandat fra FN. Norge er blant statene som har vært med fra grunnleggelsen av FN, NATO, Europarådet, OSSE og Nordisk råd, og er i tillegg til disse medlem av EØS, Verdens handelsorganisasjon, Organisasjonen for økonomisk samarbeid og utvikling og er en del av Schengen-området.

Norge er rikt på mange naturressurser som olje, gass, mineraler, tømmer, sjømat, ferskvann og vannkraft. Disse naturgitte forutsetningene har siden begynnelsen av 1900-tallet gitt landet mulighet til en rikdomsøkning få andre land har nytt glede av, og nordmenn har pr 2017 verdens tredje høyeste gjennomsnittlige inntekt, målt i BNP pr innbygger.[2] Petroleumsindustrien står for omkring 14 \% av Norges bruttonasjonalprodukt pr 2018.[3] Norge er verdens største produsent av olje og gass per capita utenfor Midtøsten. Antal sysselsatte knyttet til denne næringen sank imidlertid fra ca 232.000 i 2013 til 207.000 i 2015.[4]

Norge er rikt på mange naturressurser som olje, gass, mineraler, tømmer, sjømat, ferskvann og vannkraft. Disse naturgitte forutsetningene har siden begynnelsen av 1900-tallet gitt landet mulighet til en rikdomsøkning få andre land har nytt glede av, og nordmenn har pr 2017 verdens tredje høyeste gjennomsnittlige inntekt, målt i BNP pr innbygger.[2] Petroleumsindustrien står for omkring 14 \% av Norges bruttonasjonalprodukt pr 2018.[3] Norge er verdens største produsent av olje og gass per capita utenfor Midtøsten. Antal sysselsatte knyttet til denne næringen sank imidlertid fra ca 232.000 i 2013 til 207.000 i 2015.[4]

Norge er rikt på mange naturressurser som olje, gass, mineraler, tømmer, sjømat, ferskvann og vannkraft. Disse naturgitte forutsetningene har siden begynnelsen av 1900-tallet gitt landet mulighet til en rikdomsøkning få andre land har nytt glede av, og nordmenn har pr 2017 verdens tredje høyeste gjennomsnittlige inntekt, målt i BNP pr innbygger.[2] Petroleumsindustrien står for omkring 14 \% av Norges bruttonasjonalprodukt pr 2018.[3] Norge er verdens største produsent av olje og gass per capita utenfor Midtøsten. Antal sysselsatte knyttet til denne næringen sank imidlertid fra ca 232.000 i 2013 til 207.000 i 2015.[4]

Norge er rikt på mange naturressurser som olje, gass, mineraler, tømmer, sjømat, ferskvann og vannkraft. Disse naturgitte forutsetningene har siden begynnelsen av 1900-tallet gitt landet mulighet til en rikdomsøkning få andre land har nytt glede av, og nordmenn har pr 2017 verdens tredje høyeste gjennomsnittlige inntekt, målt i BNP pr innbygger.[2] Petroleumsindustrien står for omkring 14 \% av Norges bruttonasjonalprodukt pr 2018.[3] Norge er verdens største produsent av olje og gass per capita utenfor Midtøsten. Antal sysselsatte knyttet til denne næringen sank imidlertid fra ca 232.000 i 2013 til 207.000 i 2015.[4]

Norge er rikt på mange naturressurser som olje, gass, mineraler, tømmer, sjømat, ferskvann og vannkraft. Disse naturgitte forutsetningene har siden begynnelsen av 1900-tallet gitt landet mulighet til en rikdomsøkning få andre land har nytt glede av, og nordmenn har pr 2017 verdens tredje høyeste gjennomsnittlige inntekt, målt i BNP pr innbygger.[2] Petroleumsindustrien står for omkring 14 \% av Norges bruttonasjonalprodukt pr 2018.[3] Norge er verdens største produsent av olje og gass per capita utenfor Midtøsten. Antal sysselsatte knyttet til denne næringen sank imidlertid fra ca 232.000 i 2013 til 207.000 i 2015.[4]

Norge er rikt på mange naturressurser som olje, gass, mineraler, tømmer, sjømat, ferskvann og vannkraft. Disse naturgitte forutsetningene har siden begynnelsen av 1900-tallet gitt landet mulighet til en rikdomsøkning få andre land har nytt glede av, og nordmenn har pr 2017 verdens tredje høyeste gjennomsnittlige inntekt, målt i BNP pr innbygger.[2] Petroleumsindustrien står for omkring 14 \% av Norges bruttonasjonalprodukt pr 2018.[3] Norge er verdens største produsent av olje og gass per capita utenfor Midtøsten. Antal sysselsatte knyttet til denne næringen sank imidlertid fra ca 232.000 i 2013 til 207.000 i 2015.[4]

Norge er rikt på mange naturressurser som olje, gass, mineraler, tømmer, sjømat, ferskvann og vannkraft. Disse naturgitte forutsetningene har siden begynnelsen av 1900-tallet gitt landet mulighet til en rikdomsøkning få andre land har nytt glede av, og nordmenn har pr 2017 verdens tredje høyeste gjennomsnittlige inntekt, målt i BNP pr innbygger.[2] Petroleumsindustrien står for omkring 14 \% av Norges bruttonasjonalprodukt pr 2018.[3] Norge er verdens største produsent av olje og gass per capita utenfor Midtøsten. Antal sysselsatte knyttet til denne næringen sank imidlertid fra ca 232.000 i 2013 til 207.000 i 2015.[4]



\chapter{Sweden}
Det skal vel la seg gjøre med litt tekst før vi starter med seksjonene

\section{Swedish tostands}
Norge er et parlamentarisk demokrati og konstitusjonelt monarki, hvor Harald V siden 1991 er konge og statsoverhode, og Erna Solberg (H) er siden 2013 statsminister. Norge er en enhetsstat, med to administrative nivå under staten: fylker og kommuner. Den samiske delen av befolkningen har, gjennom Sametinget og Finnmarksloven, til en viss grad selvstyre og innflytelse over tradisjonelt samiske områder. Selv om Norge har avvist medlemskap i Den europeiske union gjennom to folkeavstemninger, har Norge gjennom EØS-avtalen tette bånd til unionen, og gjennom NATO med USA. Norge er en betydelig bidragsyter i De forente nasjoner (FN), og har deltatt med soldater i flere utenlandsoperasjoner med mandat fra FN. Norge er blant statene som har vært med fra grunnleggelsen av FN, NATO, Europarådet, OSSE og Nordisk råd, og er i tillegg til disse medlem av EØS, Verdens handelsorganisasjon, Organisasjonen for økonomisk samarbeid og utvikling og er en del av Schengen-området.

Norge er rikt på mange naturressurser som olje, gass, mineraler, tømmer, sjømat, ferskvann og vannkraft. Disse naturgitte forutsetningene har siden begynnelsen av 1900-tallet gitt landet mulighet til en rikdomsøkning få andre land har nytt glede av, og nordmenn har pr 2017 verdens tredje høyeste gjennomsnittlige inntekt, målt i BNP pr innbygger.[2] Petroleumsindustrien står for omkring 14 \% av Norges bruttonasjonalprodukt pr 2018.[3] Norge er verdens største produsent av olje og gass per capita utenfor Midtøsten. Antal sysselsatte knyttet til denne næringen sank imidlertid fra ca 232.000 i 2013 til 207.000 i 2015.[4]

Norge er rikt på mange naturressurser som olje, gass, mineraler, tømmer, sjømat, ferskvann og vannkraft. Disse naturgitte forutsetningene har siden begynnelsen av 1900-tallet gitt landet mulighet til en rikdomsøkning få andre land har nytt glede av, og nordmenn har pr 2017 verdens tredje høyeste gjennomsnittlige inntekt, målt i BNP pr innbygger.[2] Petroleumsindustrien står for omkring 14 \% av Norges bruttonasjonalprodukt pr 2018.[3] Norge er verdens største produsent av olje og gass per capita utenfor Midtøsten. Antal sysselsatte knyttet til denne næringen sank imidlertid fra ca 232.000 i 2013 til 207.000 i 2015.[4]

Norge er rikt på mange naturressurser som olje, gass, mineraler, tømmer, sjømat, ferskvann og vannkraft. Disse naturgitte forutsetningene har siden begynnelsen av 1900-tallet gitt landet mulighet til en rikdomsøkning få andre land har nytt glede av, og nordmenn har pr 2017 verdens tredje høyeste gjennomsnittlige inntekt, målt i BNP pr innbygger.[2] Petroleumsindustrien står for omkring 14 \% av Norges bruttonasjonalprodukt pr 2018.[3] Norge er verdens største produsent av olje og gass per capita utenfor Midtøsten. Antal sysselsatte knyttet til denne næringen sank imidlertid fra ca 232.000 i 2013 til 207.000 i 2015.[4]

Norge er rikt på mange naturressurser som olje, gass, mineraler, tømmer, sjømat, ferskvann og vannkraft. Disse naturgitte forutsetningene har siden begynnelsen av 1900-tallet gitt landet mulighet til en rikdomsøkning få andre land har nytt glede av, og nordmenn har pr 2017 verdens tredje høyeste gjennomsnittlige inntekt, målt i BNP pr innbygger.[2] Petroleumsindustrien står for omkring 14 \% av Norges bruttonasjonalprodukt pr 2018.[3] Norge er verdens største produsent av olje og gass per capita utenfor Midtøsten. Antal sysselsatte knyttet til denne næringen sank imidlertid fra ca 232.000 i 2013 til 207.000 i 2015.[4]

Norge er rikt på mange naturressurser som olje, gass, mineraler, tømmer, sjømat, ferskvann og vannkraft. Disse naturgitte forutsetningene har siden begynnelsen av 1900-tallet gitt landet mulighet til en rikdomsøkning få andre land har nytt glede av, og nordmenn har pr 2017 verdens tredje høyeste gjennomsnittlige inntekt, målt i BNP pr innbygger.[2] Petroleumsindustrien står for omkring 14 \% av Norges bruttonasjonalprodukt pr 2018.[3] Norge er verdens største produsent av olje og gass per capita utenfor Midtøsten. Antal sysselsatte knyttet til denne næringen sank imidlertid fra ca 232.000 i 2013 til 207.000 i 2015.[4]

Norge er rikt på mange naturressurser som olje, gass, mineraler, tømmer, sjømat, ferskvann og vannkraft. Disse naturgitte forutsetningene har siden begynnelsen av 1900-tallet gitt landet mulighet til en rikdomsøkning få andre land har nytt glede av, og nordmenn har pr 2017 verdens tredje høyeste gjennomsnittlige inntekt, målt i BNP pr innbygger.[2] Petroleumsindustrien står for omkring 14 \% av Norges bruttonasjonalprodukt pr 2018.[3] Norge er verdens største produsent av olje og gass per capita utenfor Midtøsten. Antal sysselsatte knyttet til denne næringen sank imidlertid fra ca 232.000 i 2013 til 207.000 i 2015.[4]

Norge er rikt på mange naturressurser som olje, gass, mineraler, tømmer, sjømat, ferskvann og vannkraft. Disse naturgitte forutsetningene har siden begynnelsen av 1900-tallet gitt landet mulighet til en rikdomsøkning få andre land har nytt glede av, og nordmenn har pr 2017 verdens tredje høyeste gjennomsnittlige inntekt, målt i BNP pr innbygger.[2] Petroleumsindustrien står for omkring 14 \% av Norges bruttonasjonalprodukt pr 2018.[3] Norge er verdens største produsent av olje og gass per capita utenfor Midtøsten. Antal sysselsatte knyttet til denne næringen sank imidlertid fra ca 232.000 i 2013 til 207.000 i 2015.\cite{energy_2018}

\section{What is possible to do?}
Norge er rikt på mange naturressurser som olje, gass, mineraler, tømmer, sjømat, ferskvann og vannkraft. Disse naturgitte forutsetningene har siden begynnelsen av 1900-tallet gitt landet mulighet til en rikdomsøkning få andre land har nytt glede av, og nordmenn har pr 2017 verdens tredje høyeste gjennomsnittlige inntekt, målt i BNP pr innbygger.\cite{tipler} Petroleumsindustrien står for omkring 14 \% av Norges bruttonasjonalprodukt pr 2018.[3] Norge er verdens største produsent av olje og gass per capita utenfor Midtøsten. Antal sysselsatte knyttet til denne næringen sank imidlertid fra ca 232.000 i 2013 til 207.000 i 2015.\cite{energy_2018}



Norge er rikt på mange naturressurser som olje, gass, mineraler, tømmer, sjømat, ferskvann og vannkraft. Disse naturgitte forutsetningene har siden begynnelsen av 1900-tallet gitt landet mulighet til en rikdomsøkning få andre land har nytt glede av, og nordmenn har pr 2017 verdens tredje høyeste gjennomsnittlige inntekt, målt i BNP pr innbygger.[2] Petroleumsindustrien står for omkring 14 \% av Norges bruttonasjonalprodukt pr 2018.[3] Norge er verdens største produsent av olje og gass per capita utenfor Midtøsten. Antal sysselsatte knyttet til denne næringen sank imidlertid fra ca 232.000 i 2013 til 207.000 i 2015.\cite{energy_2018}

\printbibliography
\end{document}