\documentclass[class=book, crop=false]{standalone}
\usepackage[subpreambles=true]{standalone}
\usepackage{import}
\usepackage[ruled,vlined]{algorithm2e}

\usepackage{amsmath}
\usepackage{amssymb}
\usepackage[margin=1.2in]{geometry}
\usepackage[sorting = none,
            doi = true  %lesedato for url-adresse
            ]{biblatex} %none gir bibliografi i sitert rekkefølge
\addbibresource{reference.bib}
\usepackage{csquotes}
\usepackage{pgfplots}
\pgfplotsset{compat=1.15}

\begin{document}
\section{Reinforcement learning}
Reinforcement learning is a discipline in rapid development. Reinforcement algorithms have gain from the recent success of methods in supervised learning, such as convolutional neural networks (CNN) and  recurrent neural networks (RNN). Dan Cire\c{s}an et al. had a breakthrough in 2010 when they were able to train a dense neural network using back propagation with a graphic processor unit (GPU) instead of a conventional central processor unit (CPU) \cite{DNN_gpu_2010}. The neural network was trained on the MNIST dataset of handwritten digits using image augmentation such as rotations, horizontally shearing and scaling. Their result showed that the time of the back propagation routine was cut by a factor of 40 with the use of the GPU. They achieved a record breaking low error rate of 0.35 \% in the classification task. In 2011 the team of Dan Cire\c{s}an et al. continued the development and presented an implementation of back propagation for CNN using GPU, cutting the training time from months to days \cite{CNN_gpu_2011}.

In 2013, Mnih et al. at DeepMind Technologies implemented a Q-learning algorithm using CNN as function approximators for the action-value function \cite{DQN_Mnih_et_al_2013}. They called their method Deep Q-Network (DQN) and applied it on seven of the Atari 2600 games. DQN was able beat all previous solutions in all games expect for space invaders, and achieve super-human performance in three games. DQN only learns from raw pixel input, without using low dimensional feature engineering of the input values. In addition, the same network architecture and hyper-parameters were used for all games. Before DQN, most reinforcement learning used linear function approximators for the action-values function because nonlinear approximated had problems with divergence \cite{DQN_Mnih_et_al_2013}. In addition, Tsitsiklis and Van Roy had presented a proof of convergence, in addition to a bound on the approximation error for linear approximators \cite{linear_stable_Tsitsiklis97}.  
A problem with using neural networks as a function approximator in online learning is that samples in reinforcement learning are highly correlated. States visited in chronological order are naturally dependent on each other. Therefore, it is difficult to update the network online, since the gradient estimate will suffer from large variance during training. Mnih et. al avoided this problem by using an experience replay buffer that stores the \textit{N} last tuples $\langle s_{t}, a_{t}, r_{t}, s_{t+1}\rangle$ of experiences made by the agent. The neural network is updated using stochastic gradient descent by drawing random samples from the the experience replay buffer. In addition to avoiding correlated samples, it is a more sample efficient approach as a single experience can be used in different parameter updates \cite{DQN_Mnih_et_al_2013}.


In each time step, Mnih et. al converted the last 4 from RGB to grayscale images, reduced them to 84x84 pixels, and stacked into a 84x84x4 picture. This was the state representation that was sent as input to the CNN. The output layer of the neural network was the action-value for all of the actions in that game. Note that the action space in these games all are discrete actions, i.e move right, move left, shoot etc. Until DQN, similar algorithms all relied on some sort of domain knowledge that was used for to feature engineering. DQN, on the other hand, learns to extract the relevant features on its own, from a stack of grayscale images, through its neural network function approximator.

In 2015, Mnih et. al tested DQN on more atari games, and outperformed 43 of the 47 games where reinforcement learning research previously has been conducted \cite{mnih2015_humanlevel}. The algorithm had a slight modification to that presented in the 2013 paper. The Bellman equation in \eqref{eq:theory:bellman_equation} is the foundation for calculating the target values used for updating the parameters in the function approximator. However, the parameters of the neural network are used to calculate the target values which in turn are used to calculate the parameter update. As a result, the learning situation can become unstable because the target values are changing as the network parameters are updated. To solve this, they only periodically updated the parameters used for calculating the target. 


Although DQN was a great improvement compared to existing methods, it is not able to tackle tasks with a continuous action space. A self-driving car must not only decide to turn left or right, but it must determine the exact angle to turn the wheel. A possible solution is to discretise the action space, for instance by dividing the possible wheel angles into 5 possible categories. This is realisable for tasks consisting of few independent action variables, but the action space grows exponentially with the number of variables. For instance, imagine a reinforcement task with 10 independent action variables that all should be discretised into 5 categories. The resulting action space is finite, but has a size of $5^{10} \approx 10, 000,000$. The consequence is that the final layer in the neural network in the DQN algorithm must consists of 10 million neurons, which all would have a parameter for all of the neurons in the previous layer, in addition to a bias. The number of parameters in the neural network would easily surpass 1 billion. Therefore, a better approach is to use policy gradient methods that are well suited for giving a continuous action. Lillicrap et al. presented the deep deterministic policy gradient (DDPG) as an extension of the DQN algorthm \cite{DBLP:journals/corr/LillicrapHPHETS15}.   



\end{document}