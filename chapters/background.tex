\documentclass[class=book, crop=false]{standalone}
\usepackage[subpreambles=true]{standalone}
\usepackage{import}
\usepackage[ruled,vlined]{algorithm2e}

\usepackage{amsmath}
\usepackage{amssymb}
\usepackage[margin=1.2in]{geometry}
\usepackage[sorting = none,
            doi = true  %lesedato for url-adresse
            ]{biblatex} %none gir bibliografi i sitert rekkefølge
\addbibresource{reference.bib}
\usepackage{csquotes}
\usepackage{pgfplots}

\pgfplotsset{compat=1.15}

\begin{document}
The electric transmission system is an infrastructure that is vital for the modern society as virtually everything relies on a reliable and secure supply of electric power. Statnett is the transmission system operator in Norway and is responsible for stable supply of electric power and the  maintenance of 11 000 kilometers of high voltage transmission lines. The growing share of uncontrollable renewable energy sources is changing the dynamics in the power system and offers both new challenges and opportunities. A fundamental property of an electric power grid is that the electric supply and demand must equal each other at all times. In other words, the electric power generated from all power plant, wind parks and solar farms must at all time be consumed somewhere in the grid, either by consumers or in the form of losses in the lines. The power balance is a self-fulfilling fact that always holds, but the power flow will damage the grid and electric equipment if it is not controlled appropriately. If too much power is produced, the voltages in the grid will increase such that electric devices will draw more power from the grid. It is Statnett's task to balance the production and supply at all times. Norway has a somewhat easier job in terms of system control compared to other nations because over 94 \% of the electric energy production comes from hydroelectric power plants \cite{energifakta_norge}. Hydropower plants are flexible and can change its power production faster and cheaper than thermal power plants running on nuclear, coal and gas. As renewable energy increases, the flexibility and stability in the power system goes down. Thermal and hydroelectric power plants have large spinning masses (the turbines and generators) that gives inertia to the power system. They naturally resist changes in frequency of the voltage in the grid, which is a very convenient property. This is unfortunately not a feature solar power or wind power have. As a result, the job of transmission system operator becomes more complicated as uncontrollable renewable energy grows. Because renewable energy production must increase to able to meet the Paris agreement, the transmission system operators simply must adapt to this new situation.  

A problem with distributed energy resources such as solar power production, is that peak production  normally occurs around noon, when the sun is at its highest, while peak power consumption in residential areas is in the afternoon. I some cases a consequence of this is that power during daytime must be exported out to the central grid due to excess power from solar production, and later it must be imported to meet the peak power demand. As a result, there are two periods of the day where residential consumers are highly dependent on the electrical grid. As solar production becomes more common in residential areas, the amount of power that must be exported out during daytime increases, possibly challenging available power line capacities. Similarly, the amount of power that must be imported in the afternoon can grow over time due to city growth in urban areas. A possible consequence is that the existing grid must be upgraded to support higher power flow, which is an investment that is very costly and in some sense unnecessary. The consumers could be self-sustained in terms of energy in day, but they must rely on the grid because production and consumption at all times must equal each other.

Shifting the power demand curve to overlap more with the solar power production is way cope with the export/import problem. This can be done by demand response, where consumers alter their demand profile during the day. For instance, electric vehicles, dishwashers and washing machine could consume power during peak solar production, instead of during peak demand in the afternoon. The effect is that less power must be exported and imported during a normal day, possibly avoiding a costly upgrade of transmission system infrastructure. 

The scope of this thesis is to explore the possibility of using of a reinforcement learning algorithm to control an electric power system through demand response. The algorithm used is called deep determinstic policy gradient (DDPG), and the goal of this thesis is to answer the following research question (RQ):

\begin{displayquote}
\textbf{RQ}: Is the DDPG algorithm able to reduce the number of safety violations in a grid with high peak solar power production and high peak demand by the means of demand response? 
\end{displayquote}


First, a theoretic background for electric power system and reinforcement learning will be given. Thereafter, an approach for building a reinforcement learning algorithm that controls an electric grid is presented. Lastly, the algorithm is applied on small test cases and the results are discussed.


\section{Suitable tasks for a reinforcement learning algorithm in an electric transmission system}

There are several possible task that artificial intelligence and reinforcement learning can solve in an electric transmission system. The increasing amount of renewable energy penetrating modern power systems reduces the natural inertia associated with synchronous generators. There are virtual inertia systems that behave as synchronous machines from the perspective of the transmission system operator \cite{virtual_inertia}. Simply put, the main methods consist of a feedback loop that monitors the deviations from desired frequency and feeds a signal to the power inverter so it acts in the same manner as a synchronous generator. It could be a suitable task for a reinforcement learning agent to control the signal sent to the power inverters in a power system, with the goal of assuring frequency stability.



\subsection{Managing fluctuating power production from renewable energy}

\subsection{Islanding - operating subparts of the grid}




\end{document}